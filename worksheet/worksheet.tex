\documentclass{article}
\usepackage[utf8]{inputenc}
\usepackage{xcolor}
\usepackage{mdframed}
\usepackage{tcolorbox}
\usepackage{graphicx}
\usepackage{enumerate}
\usepackage[margin=0.75in]{geometry}
\usepackage{amsmath, amsfonts}
\usepackage{hyperref}

\title{\LaTeX{} Workshop Worksheet}
\author{Theodore Nguyen}
\date{September 15th, 2022}

\begin{document}

    \section*{Exercise 1}
    \addcontentsline{toc}{section}{Exercise 1}
    \begin{mdframed}
        \begin{itemize}
            \item Local Font Size
                \begin{itemize}
                    \item Sizing changes can be contained in braces \verb|{}|
                        \begin{itemize}
                            \item {\tiny \verb|\tiny|}, {\scriptsize
                                \verb|\scriptsize|}, {\footnotesize
                                \verb|\footnotesize|}, {\small \verb|\small|}
                            \item {\large \verb|\large|}, {\Large
                                \verb|\Large|},
                                {\LARGE \verb|\LARGE|}, {\huge \verb|\huge|},
                                {\Huge \verb|\Huge|}
                            \begin{itemize}
                                \item Example: \verb|{\tiny hello}|
                                    $\rightarrow$ {\tiny hello}
                            \end{itemize}
                        \end{itemize}
                \end{itemize}
            \item Text Styling
                \begin{itemize}
                    \item Italics : \verb|\textit{hello}| $\rightarrow$
                        \textit{hello}
                    \item Bold : \verb|\textbf{hello}| $\rightarrow$
                        \textbf{hello}
                    \item Underline : \verb|\underline{hello}| $\rightarrow$
                        \underline{hello}
                \end{itemize}
        \end{itemize}
    \end{mdframed}
    \begin{mdframed}
        Introduce yourself in the following format:

        \bigskip

        {\Huge Hi}, my name is [\textbf{last name}], [\textit{first name}]. I am
        a [\underline{year name}] in the graduating \tiny{class of}
        [\tiny{year}].
    \end{mdframed}

    \section*{Exercise 2}
    \addcontentsline{toc}{section}{Exercise 2}
    \begin{mdframed}
        \begin{itemize}
            \item To go into \textbf{math mode}, type your math between a pair
                of dollar signs.
                \begin{itemize}
                    \item It is also worthwhile to include \verb|amsmath| and
                        \verb|amsfonts| packages.
                \end{itemize}
        \end{itemize}
        \begin{center}
            \begingroup
            \setlength{\tabcolsep}{8pt}
            \renewcommand{\arraystretch}{1.5}
                \begin{tabular}{| l | r | l | r |}
                    \hline
                    \verb|\leq|   & $\leq$   & \verb|\sum_{n=1}^{\infty}| & $\sum_{n=1}^{\infty}$ \\\hline
                    \verb|\geq|   & $\geq$   & \verb|\int_{a}^{b}|        & $\int_{a}^{b}$        \\\hline
                    \verb|x^2|    & $x^2$    & \verb|\frac{a}{b}|         & $\frac{a}{b}$         \\\hline
                    \verb|A_1|    & $A_1$    & \verb|\sqrt{x}|            & $\sqrt{x}$            \\\hline
                    \verb|\alpha| & $\alpha$ & \verb|\pm|                 & $\pm$                 \\\hline
                    \verb|\mu|    & $\mu$    & \verb|\sin|                & $\sin$                \\\hline
                \end{tabular}
            \endgroup
        \end{center}
    \end{mdframed}
    \begin{mdframed}
        Try typing out the following math equations/expressions:
        \begin{enumerate}[1)]
            \item $y = mx + b$
            \item $F_s = \mu_s N$
            \item $a^2 + b^2 = c^2$
            \item $\sin^2 x + \cos^2 x = 1$
            \item $\int_a^a f(x)\,dx = 0$
            \item $\sum_{n=1}^5 n = 15$
            \item \underline{Challenge:} $x = \frac{-b \pm \sqrt{b^2 -
                4ac}}{2a}$
        \end{enumerate}
    \end{mdframed}

\pagebreak

    \section*{Exercise 3}
    \addcontentsline{toc}{section}{Exercise 3}
    \begin{mdframed}
\begin{verbatim}
\begin{equation}
    e^{i \pi} + 1 = 0
\end{equation}
\end{verbatim}
    \end{mdframed}
    \begin{mdframed}
        Try using the \verb|equation| environment to type out:
        \begin{equation}
            ax^2 + bx + c = 0
        \end{equation}
    \end{mdframed}


    \section*{Exercise 4}
    \addcontentsline{toc}{section}{Exercise 4}
    \begin{mdframed}
\begin{verbatim}
\begin{align}
    x + 2x + 3x &= 12 \\
             6x &= 12 \\
              x &= 2
\end{align}
\end{verbatim}
    \end{mdframed}
    \begin{mdframed}
        Try using the \verb|align| environment to type out:
        \begin{align}
            a + 2a + 3b + 4b &= 20 \\
            3a + 3b + 4b &= 20 \\
            3a + 7b &= 20
        \end{align}
    \end{mdframed}

\end{document}
