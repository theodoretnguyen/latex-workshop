%! tex program = xelatex
\documentclass[aspectratio=43]{beamer}
\usepackage{graphicx}
\usepackage{amsmath}
\usepackage{amsfonts}
\usepackage{mdframed}
\usepackage{hyperref}

\hypersetup{colorlinks = true, linkcolor = black}

\usetheme{default}

\title{\LaTeX \ Workshop}
\author{Theodore Nguyen}
\date{September 15th, 2022}

\beamertemplatenavigationsymbolsempty
\setbeamercovered{invisible}

\begin{document}

\begin{frame}{}{}
    \maketitle
\end{frame}

\begin{frame}{A Little Bit About Me}{}
    \begin{columns}
        \column{0.5\textwidth}
            \begin{itemize}
                \item LQ Class of 2019
                \item UCLA Applied Math with a Specialization in Computing
                \item Tennis
            \end{itemize}
        \column{0.5\textwidth}
            \includegraphics[width = 2.75in, angle = 270]{me.jpeg}
    \end{columns}
\end{frame}

\begin{frame}{What is \LaTeX?}
    \begin{center}
        \begin{itemize}
            \item<2-> {\Large Typesetting system that is widely used in academia
                to create professional-looking documents}
            \item<2->
                {\Large \href{https://github.com/theodoretnguyen/latex-workshop}{Examples
                (click here)}}
        \end{itemize}
        \medskip

    \end{center}
\end{frame}

\begin{frame}{Workshop Outline}
    \tableofcontents
\end{frame}

\section{1. Creating and Using Overleaf}
\begin{frame}{1. Creating and Using Overleaf}{What is Overleaf}
\begin{itemize}
    \item Collaborative and cloud-based (think Google Docs!)
    \item Removes the need for you to install \LaTeX \ on your device
\end{itemize}
\end{frame}

\begin{frame}{1. Creating and Using Overleaf}{Setting up an Overleaf Account}
\begin{enumerate}
    \item Go to \url{www.overleaf.com} and click ``Register'' in the top right
        corner
    \item Enter an email and password
    \item Click ``Create First Project $\rightarrow$ Blank Project'' and name
        it!
\end{enumerate}
\end{frame}

\section{2. Basic \LaTeX \ Document Setup}
\begin{frame}[fragile]{2. Basic \LaTeX \ Document Setup}
\begin{center}
\begin{verbatim}
\documentclass{article}
\usepackage[utf8]{inputenc}

\title{Example}
\author{theodoretnguyen}
\date{September 2022}

\begin{document}

\maketitle

\section{Introduction}

\end{document}
\end{verbatim}
\end{center}
\end{frame}

\begin{frame}[fragile]{2. Basic \LaTeX \ Document Setup}{Breakdown}
    \begin{enumerate}
        \item<1-> \verb|\documentclass{article}|
            \begin{itemize}
                \item Type of document
            \end{itemize}
        \item<2-> \verb|\usepackage[utf8]{inputenc}|
            \begin{itemize}
                \item Use \verb|\usepackage{}| for adding packages
            \end{itemize}
        \item<3->
            \begin{verbatim}
                \title{Example}
                \author{theodoretnguyen}
                \date{September 2022}
            \end{verbatim}
            \begin{itemize}
                \item Title, author, date (preamble)
            \end{itemize}
        \item<4-> \verb|\begin{document}|
            \begin{itemize}
                \item Begins the actual document
            \end{itemize}
        \item<5-> \verb|\maketitle|
            \begin{itemize}
                \item Shows title, author, date
            \end{itemize}
        \item<6-> \verb|\section{Introduction}|
            \begin{itemize}
                \item Begins a new section with the name \verb|Introduction|
            \end{itemize}
        \item<7-> \verb|\end{document}|
            \begin{itemize}
                \item Ends the document. Nothing goes after this line.
            \end{itemize}
    \end{enumerate}
\end{frame}

\section{3. Formatting Your Document}
\begin{frame}[fragile]{3. Formatting Your Document}{Local Settings Throughout
    Your Document}
    \begin{columns}
        \column{0.75\textwidth}
        \begin{itemize}
            \item[]<1-> {\Large Local Font Size}
                \begin{itemize}
                    \item Sizing changes can be contained in braces \verb|{}|
                        \begin{itemize}
                            \item<1-> Example: \verb|{\tiny hello}|
                                $\rightarrow$ {\tiny hello}
                        \end{itemize}
                \end{itemize}
            \item[]<2-> {\Large Text Styling}
                \begin{itemize}
                    \item Italics : \verb|\textit{hello}| $\rightarrow$
                        \textit{hello}
                    \item Bold : \verb|\textbf{hello}| $\rightarrow$
                        \textbf{hello}
                    \item Underline : \verb|\underline{hello}| $\rightarrow$
                        \underline{hello}
                \end{itemize}
        \end{itemize}
        \column{0.25\textwidth}
            \includegraphics[width = 0.75in]{font-sizes.png}
    \end{columns}
\end{frame}

\begin{frame}{}{}
    \begin{center}
        {\Huge Exercise 1}
    \end{center}
\end{frame}

\section{4. Math Mode and Text Mode}
\begin{frame}[fragile]{4. Math Mode and Text Mode}{}
    \begin{itemize}
        \item<1-> The text that you have been writing in the document's body is
            in \textbf{text mode}.
        \item<1-> To go into \textbf{math mode}, type your math between a pair
            of dollar signs.
            \begin{itemize}
                \item<1-> It is also worthwhile to include \verb|amsmath| and
                    \verb|amsfonts| packages.
            \end{itemize}
        \item<1-> Example:
\begin{mdframed}
\begin{verbatim}
$E = mc^2$ is cool, but $e^{i \pi} + 1 = 0$
is cooler.
\end{verbatim}
        \begin{exampleblock}{Output:}<2->
            $E = mc^2$ is cool, but $e^{i \pi} + 1 = 0$ is cooler.
        \end{exampleblock}
\end{mdframed}
    \end{itemize}
\end{frame}

\section{5. Common Math Symbols and Commands}
\begin{frame}[fragile]{5. Common Math Symbols and Commands}{}
    {\small
    \begin{center}
        \begingroup
        \setlength{\tabcolsep}{8pt}
        \renewcommand{\arraystretch}{1.5}
            \begin{tabular}{| l | r |}
                \hline
                \verb|\leq|                & $\leq$                \\\hline
                \verb|\geq|                & $\geq$                \\\hline
                \verb|x^2|                 & $x^2$                 \\\hline
                \verb|A_1|                 & $A_1$                 \\\hline
                \verb|\alpha|              & $\alpha$              \\\hline
                \verb|\mu|                 & $\mu$                 \\\hline
                \verb|\sum_{n=1}^{\infty}| & $\sum_{n=1}^{\infty}$ \\\hline
                \verb|\int_{a}^{b}|        & $\int_{a}^{b}$        \\\hline
                \verb|\frac{a}{b}|         & $\frac{a}{b}$         \\\hline
                \verb|\sqrt{x}|            & $\sqrt{x}$            \\\hline
                \verb|\pm|                 & $\pm$                 \\\hline
                \verb|\sin|                & $\sin$                \\\hline
            \end{tabular}
        \endgroup
    \end{center}}
\end{frame}

\begin{frame}{}{}
    \begin{center}
        {\Huge Exercise 2}
    \end{center}
\end{frame}

\section{6. Useful Applications for \LaTeX}
\begin{frame}{6. Useful Applications for \LaTeX}{Detexify}
    \url{https://detexify.kirelabs.org/classify.html}

    \medskip

    See or know a symbol and don’t know what its command would be in \LaTeX?

    Draw it on your computer and Detexify will guess for you!
    \begin{center}
        \includegraphics[scale = 0.25]{detexify.png}
    \end{center}
\end{frame}

\begin{frame}{6. Useful Applications for \LaTeX}{Mathpix}
    \url{https://mathpix.com/}

    \medskip

    Take a screenshot of a \LaTeX \ PDF and this application instantly writes
    the commands out for you!
    \begin{center}
        \includegraphics[width = 4in]{mathpix.png}
    \end{center}
\end{frame}

\begin{frame}{6. Useful Applications for \LaTeX}{Mathb.in}
    \url{http://mathb.in/}

    \medskip

    If you want to share LaTeX with your friends but don’t feel like typing up
    an entire Overleaf document, use this application to share quick excerpts!
    \begin{center}
        \includegraphics[width = 2.5in]{mathbin.png}
    \end{center}
\end{frame}

\section{7. Utilizing Common Environments}
\begin{frame}[fragile]{7. Utilizing Common Environments}{Math Mode Environments:
    equation}
    \begin{itemize}
        \item<1-> \verb|equation|: one line of math
            \begin{itemize}
                \item<1-> Example:
\begin{mdframed}
\begin{verbatim}
\begin{equation}
    e^{i \pi} + 1 = 0
\end{equation}
\end{verbatim}
        \begin{exampleblock}{Output:}<2->
            \begin{equation}
                e^{i \pi} + 1 = 0
            \end{equation}
        \end{exampleblock}
\end{mdframed}
            \end{itemize}
    \end{itemize}
\end{frame}

\begin{frame}{}{}
    \begin{center}
        {\Huge Exercise 3}
    \end{center}
\end{frame}

\begin{frame}[fragile]{7. Utilizing Common Environments}{Math Mode Environments:
    align}   \begin{itemize}
        \item \verb|align|: aligns multiple lines of math using \& sign
            \begin{itemize}
                \item<1-> Example:
\begin{mdframed}
\begin{verbatim}
\begin{align}
    x + 2x + 3x &= 12 \\
             6x &= 12 \\
              x &= 2
\end{align}
\end{verbatim}
        \begin{exampleblock}{Output:}<2->
            \begin{align}
                x + 2x + 3x &= 12 \\
                         6x &= 12 \\
                          x &= 2
            \end{align}
        \end{exampleblock}
\end{mdframed}
            \end{itemize}
    \end{itemize}
\end{frame}

\begin{frame}{}{}
    \begin{center}
        {\Huge Exercise 4}
    \end{center}
\end{frame}

\section{8. Other Resources}
\begin{frame}{8. Other Resources}{}
    \begin{itemize}
        \item \href{https://www.overleaf.com/learn}{Overleaf Documentation}
        \item \href{https://katex.org/docs/supported.html}{Extensive List of
            Symbols}
        \item \href{https://www.michellekrummel.com/tutorials}{Ms. Krummel}
        \item
            \href{https://www.youtube.com/playlist?list=PL-p5XmQHB_JSQvW8_mhBdcwEyxdVX0c1T}{Luke
            Smith}
    \end{itemize}
\end{frame}

\section{9. Questions, Comments, Concerns?}
\begin{frame}[fragile]{9. Questions, Comments, Concerns?}{}
    \begin{center}
        Contact: \verb|theodoretnguyen@g.ucla.edu|
    \end{center}
\end{frame}

\end{document}
